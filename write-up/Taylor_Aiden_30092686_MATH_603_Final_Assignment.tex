\documentclass[11pt]{article}
\setlength{\textwidth}{430pt}\setlength{\oddsidemargin}{11pt}
\usepackage[utf8]{inputenc}
\usepackage{amsmath}
\usepackage{amsfonts}
\usepackage{enumerate}
\usepackage{hyperref}
\usepackage{tikz}
\usetikzlibrary{trees}
\hypersetup{
    colorlinks,
    citecolor=black,
    filecolor=black,
    linkcolor=blue,
    urlcolor=black
}
\title{MATH 603 - Final Assignment}
\author{Aiden Taylor - 30092686}
\date{December 5, 2024}
\begin{document}
\tikzstyle{level 1}=[level distance=4cm, sibling distance=7cm]
\tikzstyle{level 2}=[level distance=4cm, sibling distance=4.5cm]
\tikzstyle{level 3}=[level distance=4cm, sibling distance=4.5cm]
\tikzstyle{textarea} = [text width=9em, text centered]
%\tikzstyle{end} = [circle, minimum width=3pt,fill, inner sep=0pt]
\tikzstyle{end} = []
\maketitle
\newpage
\tableofcontents
\newpage
\section{Problems}
	\begin{enumerate}[1.]
		\item Write a computer program to implement the Fast Fourier Transform (FFT).
	\item Using the FFT, write a computer program to solve numerically
		the initial-value problem (IVP) for the heat equation,
			\[
		\begin{cases}
			u_t = u_{xx} & (t,x) \in [0,T] \times [0,1] \\
			u(0,x) = u_0(x) & x \in [0,1]
		\end{cases}.
			\]
	\end{enumerate}
\newpage
\section{Problem 1}
To implement the FFT, we should first revist the Continuous Fourier Transform of some function $f(x)$,
	\begin{equation*}
		F(\omega) = \hat{f}(\omega) = \mathcal{F}\left[f\right](\omega) = \frac{1}{2\pi}\int_{-\infty}^{\infty}{f(x)e^{-i\omega x}},
	\end{equation*}
where the function can be recovered as
	\begin{equation*}
		f(x) = \int_{-\infty}^{\infty}{\hat{f}(\omega)e^{i\omega x}}.
	\end{equation*}
Now, consider discretizing both the original and frequency domains into $n$ equally spaced points, where
	\begin{equation*}
		\begin{cases}
			\omega_m = 2\pi m/n, & m = 0,1,\dots,n-1, \\
			x_k = x_0 + k\Delta x, & k = 0,1,\dots,n-1,
		\end{cases}
	\end{equation*}
given that $x_0 = 0$ and $\Delta x = L/(n-1)$. Then, if we let $f_k = f(x_k)$ for $k = 0,1,\dots,n-1$,
we can define the Discrete Fourier Transform (DFT) as
	\begin{equation*}
		f_{m}^{\#} = \sum_{k=0}^{n-1}{f_ke^{-i\omega_mk}}, \quad m = 0,1,\dots,n-1,
	\end{equation*}
where the discretization of the function can be recovered as
	\begin{equation*}
		f_{k} = \frac{1}{n}\sum_{m=0}^{n-1}{f_{k}^{\#}e^{i\omega_mk}}, \quad k = 0,1,\dots,n-1,
	\end{equation*}
which we call the Inverse DFT (IDFT). Letting $\xi = e^{i 2\pi/n}$, we can instead represent the DFT and
IDFT respectively as the following two matrix-vector multiplications,
\begin{equation*}
	\begin{bmatrix}
		f_{0}^{\#} \\
		f_{1}^{\#} \\
		f_{2}^{\#} \\
		\vdots \\
		f_{n-1}^{\#}
	\end{bmatrix}
	=
	\begin{bmatrix}
		1 & 1 & 1 & \dots & 1 \\
		1 & \xi^{-1} & \xi^{-2} & \dots & \xi^{-(n-1)} \\
		1 & \xi^{-2} & \xi^{-4} & \dots & \xi^{-2(n-1)} \\
		\vdots & \vdots & \vdots & \vdots & \vdots \\
		1 & \xi^{-(n-1)} & \xi^{-2(n-1)} & \dots & \xi^{-(n-1)(n-1)} 
	\end{bmatrix}
	\begin{bmatrix}
		f_{0} \\
		f_{1} \\
		f_{2} \\
		\vdots \\
		f_{n-1}
	\end{bmatrix},
\end{equation*}
\begin{equation*}
	\begin{bmatrix}
		f_{0} \\
		f_{1} \\
		f_{2} \\
		\vdots \\
		f_{n-1}
	\end{bmatrix}
	=
	\frac{1}{n}
	\begin{bmatrix}
		1 & 1 & 1 & \dots & 1 \\
		1 & \xi^{1} & \xi^{2} & \dots & \xi^{(n-1)} \\
		1 & \xi^{2} & \xi^{4} & \dots & \xi^{2(n-1)} \\
		\vdots & \vdots & \vdots & \vdots & \vdots \\
		1 & \xi^{(n-1)} & \xi^{2(n-1)} & \dots & \xi^{(n-1)(n-1)} 
	\end{bmatrix}
	\begin{bmatrix}
		f_{0}^{\#} \\
		f_{1}^{\#} \\
		f_{2}^{\#} \\
		\vdots \\
		f_{n-1}^{\#}
	\end{bmatrix}.
\end{equation*}
Since both the DFT and IDFT are just $n \times n$ systems,
it follows that they both have a computational complexity of $\mathcal{O}(n^2)$.
From here, the FFT is derived from noticing redundancies in the computation of the DFT, specifically,
from noticing that $\xi$ is periodic and that certain powers of $\xi$ are equal. To illustrate this claim,
consider the system of equations generated by the DFT when $n = 4$,
\begin{equation*}
	\begin{cases}
		f_{0}^{\#} &= f_0\xi^0 + f_1\xi^0 + f_2\xi^0 + f_3\xi^0 \\
		f_{1}^{\#} &= f_0\xi^0 + f_1\xi^{-1} + f_2\xi^{-2} + f_3\xi^{-3} \\
		f_{2}^{\#} &= f_0\xi^0 + f_1\xi^{-2} + f_2\xi^{-4} + f_3\xi^{-6} \\
		f_{3}^{\#} &= f_0\xi^0 + f_1\xi^{-3} + f_2\xi^{-6} + f_3\xi^{-9}.
	\end{cases}
\end{equation*}
If we notice that $\xi^0 = \xi^{-4} = 1$, $\xi^{-2} = \xi^{-6} = -1$, $\xi^{-1} = \xi^{-9} = -i$, and $\xi^{-3} = i$,
then we can simplify the above system of equations to,
\begin{equation*}
	\begin{cases}
		f_{0}^{\#} &= (f_0 + f_1) + \xi^0(f_2 + f_3) \\
		f_{1}^{\#} &= (f_0 - f_1) + \xi^{-1}(f_2 - f_3) \\
		f_{2}^{\#} &= (f_0 + f_1) + \xi^{-2}(f_2 + f_3) \\
		f_{3}^{\#} &= (f_0 - f_1) + \xi^{-3}(f_2 - f_3),
	\end{cases}
\end{equation*}
which has reduced the number of computations from 16 multiplications and 12 additions to 4 multiplications
and 12 additions (without and doing any caching or precomputing).
This idea can be generalized when $n$ is some positive integer power of 2,
i.e. $n = 2^{\ell}$ where $\ell \in \mathbb{Z}^+$, which instead allows us to represent the DFT as
\begin{equation}
	\label{eq:1}
	f_{m}^{\#} = \sum_{k=0}^{n-1}{f_k\xi^{-mk}} = \sum_{k=0}^{\frac{n}{2}-1}{f_{2k}\xi^{-m(2k)}} + \sum_{k=0}^{\frac{n}{2}-1}{f_{2k + 1}\xi^{-m(2k+1)}},
\end{equation}
for $m = 0,1\dots,n-1$, where we are essentially just breaking up the summation into its even and odd indexed
summations. If we also notice that
\begin{equation}
	\begin{aligned}
		f_{m}^{\#} &= \sum_{k=0}^{\frac{n}{2}-1}{f_{2k}\xi^{-m(2k)}} + \sum_{k=0}^{\frac{n}{2}-1}{f_{2k + 1}\xi^{-m(2k+1)}} \\
		&= \sum_{k=0}^{\frac{n}{2}-1}{f_{2k}\xi^{-2mk}} + \xi^{-m}\sum_{k=0}^{\frac{n}{2}-1}{f_{2k + 1}\xi^{-2mk}} \\
	\end{aligned}
	\label{eq:2}
\end{equation}
for $m = 0,1,\dots,n-1$, then we can use the idea from (\ref{eq:1}) on each of the two individual summations in (\ref{eq:2}).
Applying this process recursively until each of the individual summations has just one term,
reduces the computational complexity of the DFT from
$\mathcal{O}(n^2)$ to $\mathcal{O}(n\log_2{n})$, giving us the FFT.
This reduction in complexity can be visualized by the corresponding binary tree generated by the recursive process,

\begin{tikzpicture}[grow=down, sloped]
\node[textarea] {$cn$}
    child {
        node[textarea] {$cn/2$}
            child {
                node[textarea] {$cn/4$}
		node[end, label=right:
		    {}] {}
                edge from parent
                node[above] {}
                node[below]  {}
            }
            child {
                node[textarea] {$cn/4$}
		node[end, label=right:
		    {}] {}
                edge from parent
                node[above] {}
                node[below] {}
            }
    }
    child {
        node[textarea] {$cn/2$}
            child {
                node[textarea] {$cn/4$}
		node[end, label=right:
		    {}] {}
                edge from parent
                node[above] {}
                node[below]  {}
            }
            child {
                node[textarea] {$cn/4$}
		node[end, label=right:
		    {}] {}
                edge from parent
                node[above] {}
                node[below]  {}
            }
    };
\end{tikzpicture}
\newpage
\section{Problem 2}
\end{document}
